\documentclass[aspectratio=169,xcolor=dvipsnames]{beamer}
% https://github.com/PM25/SimplePlus-BeamerTheme
\usetheme{SimplePlus}

\usepackage{hyperref}
\usepackage{graphicx} 
\usepackage{booktabs}

\usepackage[spanish]{babel}

% Portada

\title{\large Aplicación de técnicas de virtualización ligera para la evaluación de redes de comunicaciones}

\author{\small Autor: Enrique Fernández Sánchez \\
Tutor: Jose María Malgosa Sanahuja}

\institute[UPCT]{\\ Universidad Politécnica de Cartagena (UPCT) \\ \vspace{5px}
	\includegraphics[scale=0.2]{img/escudo_upct.png}
}

\date{\today}

%\logo{\includegraphics[scale=0.3]{img/etsit.png}}

\usepackage{tikz}
\logo{ 
	\begin{tikzpicture}[overlay,remember picture, inner sep=0pt,outer sep=0pt]
		\node[yshift=-20px,left=0.2cm] at (current page.31){
			\includegraphics[width=3cm]{img/etsit.png}
		};
	\end{tikzpicture}
}

\begin{document}
	% 1
	\begin{frame}
		\titlepage
	\end{frame}

	% 2
	\begin{frame}{Índice}
		\tableofcontents
	\end{frame}

	% ---------------------

	\section{Virtualización de funciones de red}
	
	\begin{frame}{Virtualización de funciones de red}
		content...
	\end{frame}
	
	% ---------------------
	
	\section{Interfaces de red virtuales en Linux}
	
	\begin{frame}{Interfaces de red virtuales en Linux}
		content...
	\end{frame}
	
	% ---------------------
	
	\section{Namespaces}
	
	\begin{frame}{Namespaces}
		content...
	\end{frame}
	
	% ---------------------
	
	\section{Virtualización ligera y contenedores}
	
	\begin{frame}{Virtualización ligera y contenedores}
		content...
	\end{frame}

	% ---------------------
	
	\section{Virtualización y evaluación de redes de comunicaciones}
	
	\begin{frame}{Evaluación de redes de comunicaciones}
		content...
	\end{frame}

	% ---------------------
	
	\section{Conclusión}
	
	\begin{frame}{Conclusión}
		content...
	\end{frame}

	\begin{frame}{Propuestas futuras}
		content...
	\end{frame}

	% ---------------------
	
	\section{Bibliografía}
	
	\begin{frame}{Bibliografía}
		content...
	\end{frame}
	
\end{document}