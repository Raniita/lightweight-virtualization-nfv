\documentclass[aspectratio=169,xcolor=dvipsnames]{beamer}
% https://github.com/PM25/SimplePlus-BeamerTheme
\usetheme{SimplePlus}

\usepackage{hyperref}
\usepackage{graphicx} 
\usepackage{booktabs}
\usepackage{courier}

\usepackage[spanish]{babel}

% Portada

\title{\large Aplicación de técnicas de virtualización ligera para la evaluación de redes de comunicaciones}

\author{\small \textit{Autor}: Enrique Fernández Sánchez \\
\textit{Tutor}: Jose María Malgosa Sanahuja}

\institute[UPCT]{\\ Universidad Politécnica de Cartagena (UPCT) \\ \vspace{5px}
	\includegraphics[scale=0.2]{img/escudo_upct.png}
}

\date{\today}

\usepackage{tikz}
\logo{ 
	\begin{tikzpicture}[overlay,remember picture, inner sep=0pt,outer sep=0pt]
		\node[yshift=-20px,left=0.2cm] at (current page.31){
			\includegraphics[width=3cm]{img/etsit.png}
		};
	\end{tikzpicture}
}

\begin{document}
	% 1
	\begin{frame}
		\titlepage
	\end{frame}

	% 2
	\begin{frame}{Índice}
		\tableofcontents
	\end{frame}
	
%	\begin{frame}{Índice de la memoria}
%	    \begin{enumerate}
%	        \item Introducción
%	        \item Virtualización de funciones de red
%	        \item Interfaces de red virtuales en Linux
%	        \item Espacio de nombres en Linux
%	        \item Virtualización ligera y contenedores
%	        \item Caso práctico: Virtualización para la simulación de redes
%	        \item Conclusiones
%	    \end{enumerate}
%	\end{frame}

    % --------------------
    \section{Introducción}
    
    \begin{frame}{Introducción}
    
        \begin{itemize}
            \item \textit{Abstract}: Guía ejemplificada para comprender los conceptos de virtualización ligera, contenedores y namespaces. 
        \end{itemize}
    
        \begin{block}{Objetivos del proyecto}
            \begin{itemize}
                \item Aprender \textbf{conceptos básicos de NFV}.
                \item \textbf{Diferenciar entre virtualización} ``dura'' y ``ligera''.
                \item Estudiar soluciones, dentro del sistema operativo Linux, que permitan realizar soluciones NFV.
                \item \textbf{Definir \textit{namespaces}}.
                \item Desgranar el \textbf{concepto de contenedor}.
                \item \textbf{Ejemplos de uso} de la virtualización ``ligera'' para la simulación de redes de telecomunicaciones.
            \end{itemize}
        \end{block}
    \end{frame}

	% ---------------------

	\section{Virtualización de funciones de red}
	
	\begin{frame}{Virtualización de funciones de red}
		\begin{itemize}
		    \item Contextualización sobre la importancia de la virtualización en las redes.
		\end{itemize}
		
		\begin{alertblock}{Problemas actuales de las operadoras de red}
		    \begin{itemize}
		        \item Saturación general en la red.
		        \item Dispositivos de red que requieren una instalación manual o intervención manual.
		        \item Problemas de gestión de las operadoras de red (ej: reducción de costes).
		    \end{itemize}
		\end{alertblock}
		
		\begin{itemize}
	        \item Nacimiento de NFV: \textit{white paper} en octubre de 2012 (WG de la ITU, 13 operadoras)
	        \item NFV como solución para los problemas de las operadoras de red.
	    \end{itemize}
	\end{frame}
	
	\begin{frame}{Virtualización de funciones de red}
	    \begin{columns}
	        % Columna 1
	        \begin{column}{0.5\textwidth}
	        \begin{exampleblock}{\textit{Classic networks}}
	            \begin{itemize}
	                \item Hardware embebido.
	                \item Cada dispositivo de red asociado a una única función de red (firewall, balanceador de carga, router...)
	                \item Software dependiente del ``vendor'' (cisco, jupyter...)
	                \item Gestión de recursos software/hardware limitada.
	            \end{itemize}
	        \end{exampleblock}
	        \end{column}
	        
	        % Columna 2
	        \begin{column}{0.5\textwidth}
	        \begin{exampleblock}{\textit{Network Functions Virtualization} (\texttt{NFV})}
	            \begin{itemize}
	                \item Hardware de carácter general.
	                \item Un mismo hardware puede tener asociadas más de una función de red.
	                \item Independencia del ``vendor''.
	                \item Gestión de recursos software/hardware total.
	            \end{itemize}
	        \end{exampleblock}
	        \end{column}
	    \end{columns}
	\end{frame}
	
	\begin{frame}{Virtualización de funciones de red}
	
	\begin{columns}
	% Columna 1
	\begin{column}{0.6\textwidth}
	    \begin{figure}[h]
            \includegraphics[width=0.95\textwidth]{img/classic_network_vs_nfv.jpg}
            \caption{Comparativa enfoque clásico contra enfoque virtualizado}
       \end{figure}
	\end{column}
	
	% Columna 2
	\begin{column}{0.4\textwidth}
	\begin{alertblock}{Ventajas de NFV}
	    \begin{itemize}
	        \item Reducción de costes.
	        \item Acelerar desarrollo de servicios para los operadores de red.
	        \item Desacoplar funciones de red de hardware específico.
	    \end{itemize}
	\end{alertblock}
	\end{column}
	\end{columns}
	
	\end{frame}
	
	\begin{frame}{SDN \& NFV}
        \begin{columns}
            \begin{column}{0.6\textwidth}
            \begin{alertblock}{\textit{Software Defined Networks} (\texttt{SDN})}
            \begin{itemize}
                \item NFV depende de SDN, y viceversa.
                \item SDN desacopla el plano de control y el plano de datos de hardware de red. 
                \item El controlador SDN es el encargado de programar los dispositivos vía software.
                \item El router solo sabe encaminar paquetes de la forma que el controlador SDN le ha programado.
            \end{itemize}
            \end{alertblock}
            
            \begin{block}{APIs implicadas en SDN}
                    Para programar bajo el paradigma SDN utilizamos:
                    \begin{itemize}
                        \item \texttt{Southbound}
                        \item \texttt{Northbound} 
                    \end{itemize}
            \end{block}
            \end{column}
            
            \begin{column}{0.4\textwidth}
                \begin{figure}[h]
                \includegraphics[width=0.95\textwidth]{img/south_north.jpg}
                \caption{Esquema y ámbito de aplicacion de las APIs}
                \end{figure}
            \end{column}
        \end{columns}
	\end{frame}
	
	\begin{frame}{Virtualización ``ligera''}
	\begin{columns}
	\begin{column}{0.4\textwidth}
	    \begin{block}{\textit{Lightweight virtualization}}
	        \begin{itemize}
	            \item Conocida como \texttt{containers}.
	            \item El propio \texttt{kernel} del SO realiza la virtualización.
	            \item Espacios de usuarios aislados entre sí.
	        \end{itemize}
	    \end{block}
	    
	    \begin{exampleblock}{Comparativa virt. ``dura''}
	        \begin{itemize}
	            \item Eliminamos el \textit{overhead}, al tener un único kernel.
	            \item Menor consumo de recursos.
	            \item Facilidad a la hora de manipular las instancias.
	        \end{itemize}
	    \end{exampleblock}
	\end{column}
	
	\begin{column}{0.6\textwidth}
	    \begin{figure}
            \includegraphics[width=1\textwidth]{img/virtualization_comparative_2.png}
            \caption{Comparative entre virt. dura y virt. ligera}
       \end{figure}
	\end{column}
	
	\end{columns}
	\end{frame}
	
	% ---------------------
	
	\section{Interfaces de red virtuales en Linux}
	
	\begin{frame}{Interfaces de red virtuales en Linux}
		\begin{itemize}
		    \item Linux permite una amplia selección de interfaces virtuales de red. Algunas de ellas son realmente interesantes para la implementación de virtualización ligera, ya que permiten comunicar las instancias aisladas.
		\end{itemize}
		
		\begin{alertblock}{Selección de interfaces de red virtuales en Linux}
		    \begin{itemize}
		        \item \textbf{Bridge}
		        \item \textbf{MAC compartida}
		        \item \textbf{VLAN 802.1q}
		        \item \textbf{VLAN 802.1ad}
		        \item \textbf{Pares ethernet virtuales}
		        \item \textbf{TUN/TAP}
		    \end{itemize}
		\end{alertblock}
	\end{frame}
	
	\begin{frame}{Pares ethernet virtuales}
	    \begin{columns}
	        \begin{column}{0.5\textwidth}
	                \begin{figure}[h]
                        \includegraphics[width=0.95\textwidth]{img/veth_ej2.png}
                        \caption{Diagrama de ejemplo de uso de pares ethernet virtuales}
                    \end{figure}
	        \end{column}
	        
	        \begin{column}{0.5\textwidth}
	            \begin{block}{Ejemplo: topología usando veth}
	                \begin{itemize}
	                    \item Dispositivos virtuales creados a pares.
	                    \item Los paquetes transmitidos en un extremos se reciben automáticamente en el otro extremo.
	                    \item Útiles para asignar un extremo de la pareja a un namespace.
	                    \item \texttt{ip link add ... type veth peer ...}
	                \end{itemize}
	            \end{block}
	        \end{column}
	    \end{columns}
	\end{frame}
	
	\begin{frame}{TUN/TAP}
	    \begin{columns}
	        \begin{column}{0.5\textwidth}
	            \begin{itemize}
	                \item Permiten conectar aplicaciones a través de un socket de red.
	                \begin{itemize}
	                    \item TUN. Transporta paquetes IP
	                    \item TAP. Transporta paquetes Ethernet.
	                \end{itemize}
	            \end{itemize}
	            
	            \begin{exampleblock}{Crear interfaz TUN/TAP}
	                \begin{itemize}
	                    \item \texttt{tunctl}
	                    \item \texttt{ip tuntap}
	                \end{itemize}
	            \end{exampleblock}
	        \end{column}
	        
	        \begin{column}{0.5\textwidth}
	                \begin{figure}[h]
                        \includegraphics[width=1\textwidth]{img/tun_vs_tap.png}
                        \caption{Comparativa en capa OSI de TUN/TAP}
                    \end{figure}
	        \end{column}
	    \end{columns}
	\end{frame}
	
	\begin{frame}{Ejemplos interfaces de red virtuales}
	    \begin{alertblock}{Ejemplos realizados en esta sección}
	        \begin{itemize}
	            \item Creación de cada una de las interfaces de red virtuales.
	            \item Aplicación que crea una interfaz TUN/TAP.
	            \item Aislamiento entre puertos usando interfaz TUN/TAP.
	        \end{itemize}
	    \end{alertblock}
	\end{frame}
	
	% ---------------------
	
	\section{Namespaces}
	
	\begin{frame}{Namespaces}
	
	    \begin{columns}
	    \begin{column}{0.6\textwidth}
	        \begin{block}{Definición namespace}
	           Característica del kernel de Linux que permite gestionar los recursos, pudiendo limitarlos a uno (o varios) procesos.
	           
	           \begin{itemize}
	               \item \textit{Objetivo}: adquirir una característica del sistema como una abstracción. Dentro del namespace tienen su propia instancia aislada del recurso global.
	           \end{itemize}
	        \end{block}
	        
	        \begin{alertblock}{Crear/acceder namespace}
	            \begin{itemize}
	                \item \texttt{unshare}
	                \item \texttt{nsenter}
	            \end{itemize}
	        \end{alertblock}
	    \end{column}
	    
	    \begin{column}{0.4\textwidth}
	        \begin{exampleblock}{¿Cuantos namespaces hay?}
		    \begin{enumerate}
		        \item UTS (\texttt{hostname})
		        \item \textbf{Mount}
		        \item Process ID
		        \item \textbf{Network}
		        \item User ID
		        \item Interprocess Communication
		        \item \textbf{Control groups}
		        \item \textbf{Time}
		    \end{enumerate}
		\end{exampleblock}
	    \end{column}
	    \end{columns}
	\end{frame}

% Eliminar transparencia
%    \begin{frame}{UTS - Process ID - User ID - IPC}
%        \begin{columns}
%        \begin{column}{0.5\textwidth}
%            \begin{block}{\centering Unix Time Sharing (UTS)}
%                    \begin{itemize}
%                        \item Controla el hostname asociado al proceso.
%                        %\item Uso: \\ \texttt{unshare -u /bin/sh \\ hostname <hst>}
%                    \end{itemize}
%            \end{block}
%            
%            \begin{block}{\centering Interprocess Communication (IPC)}
%                    \begin{itemize}
%                        \item Controla la comunicación entre procesos.
%                        \item Aplicación más compleja de los namespaces.
%                        \item Uso común en bases de datos.
%                    \end{itemize}
%            \end{block}
%        \end{column}
%        
%        \begin{column}{0.5\textwidth}
%            \begin{block}{\centering Process ID}
%                    \begin{itemize}
%                        \item Aísla el namespace de la ID del proceso asignado
%                        \item Dentro del namespace, el proceso piensa que tiene el ID ``1''
%                    \end{itemize}
%            \end{block}
%            
%            \begin{block}{\centering User ID}
%                    \begin{itemize}
%                        \item Permite ``mappear'' un UID/GID dentro del namespace, diferente al del host.
%                        \item Aunque dentro del namespace tengamos UID 0 (root), no tenemos acceso a los archivos root del host.
%                    \end{itemize}
%            \end{block}
%        \end{column}
%        \end{columns}
%    \end{frame}
    
    \begin{frame}{Mount namespace}
        \begin{itemize}
            \item Primer namespace en aparecer en el kernel
            \item \textit{Objetivo}: restringir la visualización de la jerarquía global de archivos. Cada namespace tiene su propio ``set'' de puntos de montaje
        \end{itemize}
        
        \begin{alertblock}{\textit{Shared subtrees}}
        Por defecto, los mount namespaces tienen activada la opción de \texttt{shared subtree}. Esta opción determina como se van a propagar los montajes a otros puntos de montaje.
        \begin{itemize}
            \item \texttt{shared mount}. Replicado, todas las copias iguales.
            \item \texttt{slave mount}. Montaje tipo esclavo, solo se propagan hacia él.
            \item \texttt{private mount}. No permite propagación.
            \item \texttt{unbindeable mount}. No permite ser asociado.
        \end{itemize}
        \end{alertblock}
    \end{frame}
    
    \begin{frame}{Mount namespace}
        \begin{columns}
            \begin{column}{0.5\textwidth}
                \begin{itemize}
                    \item Un mount namespace nos \textbf{permite modificar un sistema de archivos} en concreto, \textbf{sin que otros procesos puedan ver y/o acceder} a dicho sistema de archivo.
                    \item Permite tener \textbf{diferentes vistas para cada namespace}.
                \end{itemize}
            \end{column}
        
            \begin{column}{0.5\textwidth}
                \begin{alertblock}{Montaje tipo bind}
                    Opción de \texttt{mount} (\texttt{--bind}) que nos permite realizar un montaje de un dispositivo sobre un directorio de nuestro sistema de archivos.
                \end{alertblock}
                
                \begin{alertblock}{tmpfs}
                    Funcionalidad que permite crear un sistema de archivos temporal dentro de nuestro sistema. El sistema creado se aloja en memoria.
                \end{alertblock}
            \end{column}
        \end{columns}
    \end{frame}
    
    \begin{frame}{Network namespace}
        \begin{itemize}
            \item Permite aislar la red de un proceso.
            \item El namespace crea una interfaz virtual, con un stack de red completo.
        \end{itemize}
        
        \begin{exampleblock}{Creación de network namespace}
            \begin{itemize}
                \item Para que el network namespace sea persistente, lo tenemos que crear con: \texttt{ip netns add <nombre>} o \texttt{unshare --net=<file>}
                \item Si quisiéramos comprobar los network namespaces existentes, ejecutamos: \texttt{ip netns} o \texttt{lsns}
                \item Para tener conectividad con el exterior, tenemos que añadirle una interfaz virtual: \texttt{veth}, \texttt{ip link set <veth link> netns <nombre>}
            \end{itemize}
        \end{exampleblock}
    \end{frame}
    
    \begin{frame}{Control groups \& time}
        \begin{columns}
            \begin{column}{0.5\textwidth}
                \begin{block}{\centering Control groups (cgroups)}
                    \begin{itemize}
                        \item Mecanismos de control para los recursos del sistema.
                        \item Funcionalidades que puede realizar:
                        \begin{itemize}
                            \item Limitar recursos
                            \item Priorizar tareas
                            \item Monitorización
                            \item Control
                        \end{itemize}
                        \item Configurado utilizando sistema de archivos virtual: \texttt{/sys/fs/cgroup}
                        \item Disponibles dos versiones: cada vez más aceptada la \texttt{v2}
                    \end{itemize}
                \end{block}
            \end{column}
        
            \begin{column}{0.5\textwidth}
                \begin{block}{\centering Time}
                    \begin{itemize}
                        \item Último namespace añadido al kernel
                        \item Permite crear desfases entre los relojes del sistema.
                        \item Cambiar la fecha y hora dentro del namespace sin modificar la del host.
                    \end{itemize}
                \end{block}
            \end{column}
        \end{columns}
    \end{frame}
	
	\begin{frame}{Ejemplos namespaces}
	    \begin{alertblock}{Ejemplos realizados en esta sección}
	        \begin{columns}
	        \begin{column}{0.5\textwidth}
	            \begin{itemize}
	                \item Uso de UTS namespace
	                \item \textbf{Persistencia de un namespace}
	                \item Mount namespace con dispositivo físico
	                \item \textbf{Casos de uso de montaje tipo bind}
	                \item \textbf{Casos de uso de shared subtree}
	                \item \textbf{Topología de red usando network namespaces y NAT}
	            \end{itemize}
	        \end{column}
	        
	        \begin{column}{0.5\textwidth}
	            \begin{itemize}
	                \item Uso de process ID namespace
	                \item Uso de UID namespace
	                \item \textbf{Limitar memoria RAM con cgroups}
	                \item \textbf{Limitar uso de CPU con cgroups}
	                \item Topología de red usando comando \texttt{ip}
	                \item Topología de red usando comando \texttt{unshare}
	            \end{itemize}
	        \end{column}
	    \end{columns}
	    \end{alertblock}
	\end{frame}
	
	% ---------------------
	
	\section{Virtualización ligera y contenedores}
	
	\begin{frame}{Virtualización ligera y contenedores}
		\begin{block}{Definición virt. ligera}
		    Tipo de virtualización que se realiza a \textbf{nivel de sistema operativo}, permitiendo la \textbf{coexistencia de diferentes espacios aislados} entre sí. Todas las instancias aisladas \textbf{utilizan el mismo kernel}. 
		\end{block}
		
		\begin{block}{Definición contenedor}
		    Abstracción alto nivel de un sistema aislado creado utilizando \texttt{namespaces} y \texttt{cgroups}. Además, proporcionan unas APIs de alto nivel para interactuar con el contenedor. Algunos ejemplos de sistemas de contenedores son:
		    
		    \begin{itemize}
		        \item \texttt{LXC}, LinuX Containers.
		        \item \texttt{Docker}
		        \item \texttt{Podman}
		    \end{itemize}
		\end{block}
	\end{frame}
	
	\begin{frame}{Contenedor DIY}
	    \begin{itemize}
	        \item Diferentes \texttt{runtimes} en función de la solución de contenedores a utilizar.
	        \item Fundación \textit{Linux Foundation's Container Initiative} (\texttt{OCI}) recoge las directivas para crear \texttt{images} y \texttt{runtimes}
	        
	        \begin{columns}
	                \begin{column}{0.5\textwidth}
	                \begin{exampleblock}{\texttt{Images}}
	                    \begin{itemize}
	                        \item Contiene un sistema de archivos root con las dependencias de la aplicación.
	                        \item Imágenes diferenciadas en capas, que serán montadas por el \texttt{runtime} usando \textit{montajes de unión}.
	                        \item Capa inferior inmutable y capa superior con todo lo referido a la aplicación.
	                    \end{itemize}
	                \end{exampleblock}
	                \end{column}
	                
	                \begin{column}{0.5\textwidth}
	                \begin{exampleblock}{\texttt{Runtimes}}
	                    \begin{itemize}
	                        \item Dada una imagen, \texttt{runtime} la ejecutará.
	                        \item Creación y configuración de los namespaces (netns, pid, ipc, uts, etc).
	                        \item Configuración de cgroups para monitorización y limites de recursos.
	                        \item Montaje de las ``capas'' del sistema de archivos.
	                    \end{itemize}
	                \end{exampleblock}    
	                \end{column}
	        \end{columns}
	    \end{itemize}
	\end{frame}
	
	\begin{frame}{LXC}
	    \begin{columns}
	        \begin{column}{0.6\textwidth}
	            \begin{itemize}
	                \item Contenedores de bajo nivel. Uso de \texttt{mount namespaces} para conseguir una estructura de directorios propia de una distribución de Linux.
	                \item Virtualización de una distribución completa.
	                \item Permite crear contenedores sin privilegios, es decir, no pueden realizar acciones sobre el host.
	                \item Apto para la ejecución de entornos gráficos.
	                \item Galería de imágenes: \url{https://uk.lxd.images.canonical.com/}
	            \end{itemize}
	        \end{column}
	        
	        \begin{column}{0.4\textwidth}
	            \begin{figure}[h]
                \includegraphics[width=0.8\textwidth]{img/lxc_logo.png}
                \caption{Logotipo \texttt{LXC}}
                \end{figure}
	        \end{column}
	    \end{columns}
	\end{frame}
	
	\begin{frame}{Docker}
	    \begin{figure}[h]
        \includegraphics[width=0.6\textwidth]{img/architecture-docker.png}
        \caption{Arquitectura Docker}
        \end{figure}
	    
	    \begin{itemize}
	        \item Solución muy famosa de contenedores. Escrita en \texttt{Go}. Uso de \texttt{namespaces}.
	        \item Arquitectura cliente-servidor. Cliente se comunica con ``servicio'' para crear, ejecutar y distribuir los contenedores. 
	    \end{itemize}
	\end{frame}

    \begin{frame}{Ejemplos virtualización ligera}
        \begin{alertblock}{Ejemplo realizados sobre contenedores DIY}
            \begin{itemize}
                \item Análisis de imagen basada en \texttt{alpine}.
                \item Análisis del \texttt{runtime} para ejecutar aplicación usando imagen basada en \texttt{alpine}.
            \end{itemize}
        \end{alertblock}
        
        \begin{alertblock}{Ejemplos realizados sobre LXC}
	        \begin{itemize}
	            \item Comprobar los namespaces de un contendor LXC en Ubuntu
	        \end{itemize}
	    \end{alertblock}
    
         \begin{alertblock}{Ejemplos realizados sobre Docker}
	        \begin{itemize}
	            \item Primeros pasos
	            \item Crear un Dockerfile
	            \item Comprobar los namespaces asociados a un contenedor Docker
	        \end{itemize}
	    \end{alertblock}
    \end{frame}
    
	% ---------------------
	
	\section{Caso práctico: Virtualización para la simulación de redes}
	
	\begin{frame}{Virtualización para simulación de redes}
		\begin{itemize}
		    \item \textit{Objetivo}: evaluar y/o simular situaciones concretas de nuestras redes. Modelamos los nodos utilizando contenedores.
		\end{itemize}
		
		\begin{block}{Opciones para la evaluación y/o simulación de redes}
		    \begin{itemize}
		        \item \texttt{Shell script} para desplegar los namespaces necesarios para la topología. Uso de comandos \texttt{nsenter} y \texttt{unshare}.
		        
		        \item Desplegar e interconectar contenedores LXC. Cada contenedor con una función de red asociada.
		        
		        \item Utilizar un simulador de redes basado en namespaces. Mininet es un simulador que permite crear una red virtual realista utilizando namespaces. Permite un sencillo uso y un despliegue rápido.
		    \end{itemize}
		\end{block}
	\end{frame}
	
	\begin{frame}{Mininet}
	    \begin{itemize}
	        \item Permite customizar y desplegar topologías de red fácilmente.
	        \item Python API muy bien documentada.
	        \item Herramienta muy utilizada en el campo de la investigación, sobre todo en el ámbito de SDN.
	        \item Permite la integración de controladores SDN externos.
	    \end{itemize}
	
	    \begin{alertblock}{Ejemplos realizados con mininet}
	        \begin{itemize}
	            \item Primeros pasos
	            \item Topología simple sin mininet
	            \item Topología simple con mininet
	            \item Limitación de recursos, tanto en nodos como en enlaces
	            \item Controlador SDN externo: Ryu
	        \end{itemize}
	    \end{alertblock}
	\end{frame}
	
	\begin{frame}{Containernet}
	
	    \begin{columns}
	        \begin{column}{0.6\textwidth}
	        \begin{itemize}
	            \item Fork de \texttt{mininet}
	            \item Permite añadir a la topología contenedores Docker como hosts
	            \item Cambios dinámicos a la topología (añadir, conectar y eliminar contenedores)
	            \item Ejecución de comandos dentro de los contenedores usando Python API
	            \item Limitar uso de recursos de los contenedores
	        \end{itemize}
	        
	        \begin{alertblock}{Ejemplos realizados con containernet}
	        \begin{itemize}
	            \item Instalación y primeros pasos
	            \item Ejecución topología simple (dos contenedores, dos switchs y un controlador)
	        \end{itemize}
	        \end{alertblock}
	        \end{column}
	        
	        \begin{column}{0.4\textwidth}
	        \begin{figure}[h]
            \includegraphics[width=0.7\textwidth]{img/containernet_logo.png}
            \caption{Logotipo \texttt{containernet}}
            \end{figure}
	        \end{column}
	    \end{columns}
	\end{frame}

	% ---------------------
	
	\section{Conclusiones}
	
	\begin{frame}{Conclusiones}
		\begin{itemize}
		    \item Alcanzados todos los objetivos propuestos
		    \item Se han realizado 28 ejemplos sobre técnicas de virt. ligera
		    \item Se han aportado dos soluciones para la evaluación de redes de comunicaciones
		    \item Se ha profundizado en el concepto de contenedor y namespaces
		\end{itemize}
		
		\begin{exampleblock}{Propuestas futuras}
		    \begin{itemize}
		        \item Profundizar en las diferencias de rendimiento entre una virt. ``ligera'' y virt. ``dura''.
		        \item Investigar sobre modificaciones del protocolo OpenFlow. Además, indagar sobre DPDK sus ventajas.
		        \item Pruebas de concepto del lenguaje de programación P4.
		    \end{itemize}
		\end{exampleblock}
	\end{frame}

	% ---------------------
	
	\section{Bibliografía}
	
	\begin{frame}{Bibliografía}
		\begin{itemize}
		    \item La contenida en la memoria del proyecto: páginas 100 - 102
		\end{itemize}
	\end{frame}
	
	\begin{frame}
	    \begin{center}
	        \vspace{30px}
	        \Large Muchas gracias por su atención \\
	        \vspace{30px}
	        \large ¿Preguntas? \\
	        \vspace{60px}
	        Enlace al proyecto: \small \url{https://github.com/Raniita/lightweight-virtualization-nfv}
	    \end{center}
	\end{frame}
	
\end{document}